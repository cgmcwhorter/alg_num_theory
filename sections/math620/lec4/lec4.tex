% !TEX root = ../../../alg_num_theory.tex
% 9 - 21 - 2009
\newpage
\subsection{Discriminants\label{sec:620_4}}

% !TEX root = ../../ant_notes.tex
%9-21
\newpage
\section{ANT I, Lecture 4: Discriminants}

Let $A$ be an integral domain and $A$ is integrally closed in $F=\Frac A$. Let $L/F$ be a finite separable extension of fields. Let $A'$ denote the integral closure of $A$ in $L$. Take a basis $\{w_i\}_{i=1}^d$ for $L/F$. 

\begin{dfn}[Discriminant]
$\Disc(\{w_i\}_{i=1}^d)=\det(\Tr{L/F}(w_iw_j))$. This is in $F$.
\end{dfn}

\begin{lem}
Suppose $\{\sigma_l\}_{l=1}^d$ are the embeddings of $L$ into $\ov{F}$ over $F$. Then $\Disc(\{w_i\}_{1}^d)=\det((\sigma_l(w_i))_{i,l})^2$.
\end{lem}

\pf Define $M=(\sigma_l(w_i))_{i,l}$.
	\[
	M=
	\begin{pmatrix}
	\sigma_1(w_1) & \sigma_2(w_1) & \cdots & \sigma_d(w_1) \\
	\sigma_1(w_2) & \sigma_2(w_2) & \cdots & \sigma_d(w_2) \\
	\vdots &  & \ddots & \vdots \\
	\sigma_1(w_d) & \sigma_2(w_d) & \cdots & \sigma_d(w_d)
	\end{pmatrix}
	\]
The transpose of this is
	\[
	M^T=
	\begin{pmatrix}
	\sigma_1(w_1) & \sigma_1(w_2) & \cdots & \sigma_1(w_d) \\
	\sigma_2(w_1) & \sigma_2(w_2) & \cdots & \sigma_2(w_d) \\
	\vdots &  & \ddots & \vdots \\
	\sigma_d(w_1) & \sigma_d(w_2) & \cdots & \sigma_d(w_d)
	\end{pmatrix}
	\]
Then $MM^T=(\Tr(w_iw_j))_{i,j}$. Then $\det M^2=\det(MM^T)=\Disc(\{w_i\})$. \qed \\

\begin{ex}
Suppose $\{w_i\}_{i=1}^d= \{\alpha^{i-1}\}_{i=1}^d$, the power basis. Call $\alpha_l=\sigma_l(\alpha)$. Then $M=(\sigma_l(w_i))_{i,l}$ and $w_i=\alpha_{i-1}$. Then
	\[
	M=(\sigma_l(w_i))_{i,l}=
	\begin{pmatrix}
	\sigma_1(1) & \sigma_2(1) & \cdots & \sigma_d(1) \\
	\sigma_1(\alpha) & \sigma_2(\alpha) & \cdots & \sigma_d(\alpha) \\
	\vdots & & \ddots & \vdots & \\
	\sigma_1(\alpha^{d-1}) & \cdots & & \sigma_d(\alpha^{d-1})
	\end{pmatrix}=
	\]
Of course, this is just the Vandermonde determinant: $\det M= \prod_{1 \leq i < j \leq d} (\alpha_j-\alpha_i)$. Then we have
	\[
	\begin{split}
	\Disc(\{\alpha_i\}_{i=1}^d)&=\det((\sigma_l(\alpha_i)_{i,l})^2 \\
	&= (\prod_{1\leq i<j\leq d} (\alpha_j - \alpha_i )^2 \\
	&= (-1)^{d(d-1)/2} \prod_{i \neq j} (\alpha_i - \alpha_j) \\
	&= (-1)^{d(d-1)/2} \Nm{L/F}(f'(\alpha)) \\
	&=(-1)^{d(d-1)/2} f'(\alpha_1)f'(\alpha_2) \cdots f'(\alpha_n)
	\end{split}
	\]
where $f$ is the irreducible polynomial for $\alpha$. This follows since $f=\prod_i (x-\alpha_i)$ and $f'= \prod_{i \neq j} (x-\alpha_i)$ so that $f'(\alpha_i)= \prod_{i \neq j} (\alpha_i - \alpha_j)$. Then
	\[
	\Disc(\{\alpha^i\}_{i=0}^d)= (-1)^{d(d-1)/2} \Nm{L/F}(f'(\alpha)) 
	\]
\end{ex}

\begin{cor}
$\Disc(\{w_i\}_{1}^d)=\det((\sigma_l(w_i))_{i,l})^2$ implies that this is nonzero if and only if $\{\sigma(w_i)\}$ is a basis and if there is a change of basis to $\{w_i'\}$ with chance of basis matrix $T$, then $\Disc(\{w_i'\})= \det(T)^2 \Disc(\{w_i\})$. 
\end{cor}

\begin{cor}
\begin{enumerate}[(a)]
\item If $T \in \GL_d(A)$, where $A \subseteq F= \Frac(A)$, then 
	\[
	\Disc(\{w_i'\}) \in (U(A))^2 \Disc(\{w_i\}_i)
	\]
where $U(A)$ denotes the units of $A$.
\item Suppose $\{w_i\}$ is a basis for $L/F$ (so $\Disc(\{w_i\}_i) \neq 0$), then if $w_i'=T(w_i)$ for $T \in \GL_d(F)$
	\[
	\bigoplus Aw_i'= \bigoplus Aw_i \Longleftrightarrow T \in \GL_d(A) \Longleftrightarrow \det T=u \in U(A) \Longleftrightarrow \det(T)^2 \in U(A)^2 \Longleftrightarrow \dfrac{\Disc(\{w_i'\})}{\Disc(\{w_i\})} \in U(A)^2
	\]
When $A$ is integrally closed, this happens if and only if $\frac{\Disc(\{w_i'\})}{\Disc(\{w_i\})} \in U(A)$.
\item Say $A$ is integrally closed in $F$, $A'$ the integral closure of $A$ in $L$, then if $\{w_i\}$ is a basis contained in $A'$, $\Disc(\{w_i\}) \in A$.
\item Suppose $A$ PID, then $A'$ is a finitely generated $A$-module. But finitely generated $A$-modules are free (since there is no torsion as its sitting inside the field $L$ and $A$ is a PID). Then $A'$ has a basis over $A$. Then for all bases $\{w_i'\}$ contained in $A'$, $\Disc(\{w_i'\})\in A$.
\end{enumerate}
\end{cor}


\begin{dfn}
Let $A$ be an integral domain and $A$ is integrally closed in $F=\Frac A$. Let $L/F$ be a finite separable extension of fields. Let $A'$ denote the integral closure of $A$ in $L$. Let $D(A'/A)$ be the $A$-ideal generated by discriminants associated to bases inside $A'$. Note that if $A$ is a PID, then ideal is principal and any generator of this ideal is given by the discriminant of an $A$-basis $\{w_i\}$ of  $A'$.
\end{dfn}


\begin{dfn}[Unramified]
We say that $A'/A$ is unramified if $D(A'/A)=A$.
\end{dfn}

Note that this does not guarantee the existence of a basis $\{w_i\}$ inside $A'$ with discriminant a unit. It merely says that varying your choice of basis, you can generate all of $A$. \\


\noindent\textbf{Localizations:} Say $S$ is a multiplicatively closed subset of $A$. We define $S^{-1}A$ to be the localization. Observe that
	\[
	\begin{tikzcd}
	S^{-1}A' \arrow[dash]{d} \\
	S^{-1}A
	\end{tikzcd}
	\]	
Now localizations behave well under discriminants: $D(S^{-1}S'/S^{-1}A)=S^{-1}D(A'/A)$.
	\[
	\begin{tikzcd}
	 A' \arrow[draw=none]{r}[sloped,auto=false]{\supseteq} & A[x] \arrow[draw=none]{d}[sloped,auto=false]{\supseteq} \arrow[draw=none]{r}[sloped,auto=false]{\subseteq} & L=F(\alpha) \arrow[dash]{d} \\
	& A \arrow[draw=none]{r}[sloped,auto=false]{\subseteq} & F=\Frac A
	\end{tikzcd}
	\] 
	\[
	\Disc(\{\alpha^i\}_{i=1}^d)= (-1)^{d(d-1)/2} \Nm{L/F}(f'(\alpha)) \in (A) \wedge \neq 0
	\]
Now consider localization. Specifically, localize at $\alpha$: $S=\{\alpha^i\}$. So we are inverting the norm, i.e. this discriminant. Sometimes in this case we denote $S^{-1}A=A[1/\alpha]$. Now $A[1/\alpha] \subseteq A'[1/\alpha]$. Now if we let $y= f'(\alpha)$, then $A[\alpha][1/y]$ is the integral closure of $A[1/y]$ in $L$ and is unramified over $A[1/y]$. Also, the branch locus of $\spec A' \to \spec A$ is contained inside $\spec(A/Ay) \subseteq_{\text{closed}} \spec A$.


\begin{dfn}[Principal Discriminant]
Suppose $A$ is an integral domain and $A$ is integrally closed in $F=\Frac A$. Let $L/F$ be a finite separable extension of fields. Let $A'$ denote the integral closure of $A$ in $L$. 
	\[
	\begin{tikzcd}
	 A' \arrow[draw=none]{d}[sloped,auto=false]{\supseteq} \arrow[draw=none]{r}[sloped,auto=false]{\subseteq} & L \arrow[dash]{d} \\
	A \arrow[draw=none]{r}[sloped,auto=false]{\subseteq} & F=\Frac A
	\end{tikzcd}
	\] 
Now $D(A'/A)$ is the $A$--ideal generated by all $\Disc(\{w_i\}_i)$ of $\{w_i\}_{i=1}^d \subseteq A'$. However, we could define $D_{\text{prin}}(A'/A)$ the $A$--ideal generated by the power basis associated to elements of $A$, i.e. $\Disc(\{\alpha^{i-1}\}_{i=1}^d)$ for $\alpha \in A'$. This is not always the same as the discriminant, even if you look at all ideal generated by all discriminants of power bases and you allow the power basis to change. In general, $D_{\text{prin}}(A'/A) \subseteq D(A'/A)$.
\end{dfn}

We know that if $A$ is a PID, there exists a basis $\{w_i\} \subseteq A'$ such that $D(A'/A)$ is the ideal generated by the powers of the $w_i$. ($A'$ is free as an $A$-module on some basis, take that basis, and look at its discriminant and that will generate the discriminant ideal). In this case, is there always an $\alpha \in A'$ so that $D_{\text{prin}}(A'/A)$ is generated by that basis? This seems to be an open question, even in the case $A=\Z$. 

\begin{rem}
You should try all of this lecture for the case of a quadratic extensions of $\Q$. 
\end{rem}

A useful fact is that suppose $A=\Z \subseteq F=\Q$, $L/\Q$ is a number field and $A'=\co_L \supseteq \{w_i\}_{i=1}^d\}$ with $|\Disc(\{w_i\})|$ is minimal among all $|\Disc(\{w_i'\})| \neq 0$ associated to $\{w_i'\} \subseteq \co_L$, then $\co_L= \oplus_{i=1}^d \Z w_i$. This is so if $|\Disc(\{w_i'\})|$ is a square-free integer. [If you find a lot of elements which are integral over $L$ so that the discriminant is square free, then if that weren't a basis there would be another basis so that the found discriminant would be the determinant of a determinant of a matrix in $\Z$ squared times the discriminant of the found basis. So the given discriminant would have a square factor.] This is a nice condition but it is only sufficient and not necessarily necessary. Note that if $L$ is quadratic and $d$ is a square-free integer, then $\Disc(L/\Z)$ is either $d$, if $d \equiv 1 \mod 4$ or $4d$ otherwise. 