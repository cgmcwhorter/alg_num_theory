% !TEX root = ../../../alg_num_theory.tex
% 9 - 23 - 2009
\newpage
\subsection{Cyclotomic Integers, Kronecker-Weber, Drinfeld Modules, Dedekind Rings\label{sec:620_5}}

Now we shall see some more examples of integral closures:

\begin{thm}
Suppose $p$ is a prime and $n \geq 1$. Consider the $p^n$th cyclotomic polynomial, $\Phi_{p^n}(x)= \dfrac{x^{p^n}-1}{x^{p^{n-1}}-1}= 1+x^{p^{n-1}}+ x^{2p^{n-1}}+ \cdots + x^{(p-1)p^{n-1}}$, then $\Phi_{p^n}(x)$ is irreducible of degree $\phi(p^n)=\#(\Z/p^n)$ in $\Q[x]$. Let $\zeta=\zeta_p$ be a root in some $\overline{\Q}$. Then $L=Q(\zeta)$ has degree $\phi(p^n)$ and integers $\co_L=\Z[\zeta]$. 
\end{thm}

\pf If $\zeta$ has order exactly $p^n$ in $\overline{\Q}$, for $i,j \in \Z$ prime to $p$, we can find $k \in \Z$ so that $i \equiv jk \mod p^n$ so $\dfrac{1-\zeta^i}{1-\zeta^j}=\dfrac{1-\zeta^{jk}}{1-\zeta^j}=1+\zeta^j+\cdots+\zeta^{j(k-1)} \in \Z[\zeta] \subseteq \co_L$. Similarly, $\dfrac{1-\zeta^j}{1-\zeta^i} \in \Z[\zeta]$ so that $\dfrac{1-\zeta^i}{1-\zeta^j} \in \Z[\zeta]^* \subseteq \co_L^*$ is a unit. Finite subgroups of $\Q^*$ are cyclic and observe
	\[
	\Phi_{p^n}(x)= \prod_{\substack{(i,p)=1 \\ 1 \leq i \leq \phi(p^n)=p^n-1}} (x- \zeta^i)
	\]
So we look at $\Phi(1)$ two ways, using the initial formula and the product formula bove. This gives us
	\[
	1+1+\cdots+1=p=\Phi_{p^n}(1)=\prod_{\substack{(i,p)=1 \\ 1 \leq i \leq \phi(p^n)=p^n-1}} (1- \zeta^i)
	\]
But each element in the product is equivalent to a unit times $1-\zeta$ so that the product is $u(1-\zeta)^{\phi(p^n)}$ for some unit $u \in \Z[\zeta]^* \subseteq \co_L^*$. Now $p\Z[\zeta]$ we do not know the rank of over $\Z$ but it is a finitely generated abelian group and $p\Z[\zeta]=(1-\zeta)^{\phi(p^n)} \Z[\zeta]$. But then we have
	\[
	p\Z[\zeta]=(1-\zeta)^{\phi(p^n)}\Z[\zeta] \subseteq (1-\zeta)^{\phi(p^n)-1} \Z[\zeta] \subseteq \cdots \subseteq (1-\zeta)\Z[\zeta] \subseteq \Z[\zeta]
	\]
Here $(1-\zeta)^{i+1} \Z[\zeta] \neq (1-\zeta)^i \Z[\zeta]$ since if we had equality, $(1-\zeta)\Z[\zeta]=\Z[\zeta]$ for then $1-\zeta \in \Z[\zeta]^*$, a contradiction. 
	\[
	(1-\zeta)^{\phi(p^n)} \Z[\zeta] = p\Z[\zeta] \neq \Z[\zeta]
	\]
But then (noting $\Z[\zeta]$ is a finitely generated abelian group so it has some rank--it's a direct sum of copies of $\Z$ so taking quotients obtains a sum of copies of $\dim_{\Z/p} \Z[\zeta]/(p\Z[\zeta]) = \rank_\Z \Z[\zeta]$) from the above chain, we have at least one factor of $p$ coming in as one goes up the chain showing $\phi(p^n) \leq \dim_{\Z/p} \Z[\zeta]/(p\Z[\zeta]) = \rank_\Z \Z[\zeta]$. But  then $\phi(p^n) \leq \dim_{\Z/p} \Z[\zeta]/(p\Z[\zeta]) = \rank_\Z \Z[\zeta] \leq \rank_\Z \co_L= \dim_\Q L$, where the last equality follows from the fact that there is a basis for $\co_L$ that consists of a basis for the field elements over $\Q$. Now $\dim_\Q L = \dim_\Q \Q(\zeta)$ since $\zeta$ is a generator. But this the degree of the minimal polynomial which is at most the degree of $\Phi$. Therefore, we have shown
	\[
	\phi(p^n) \leq \dim_{\Z/p} \Z[\zeta]/(p\Z[\zeta]) = \rank_\Z \Z[\zeta] \leq \rank_\Z \co_L= \dim_\Q L = \dim_\Q \Q(\zeta) \leq \deg \Phi_{p^n}(x)=\phi(p^n)
	\]
This shows $[L:\Q]=\dim_\Q L= \phi(p^n)$ and $\phi_{p^n}(x)$ is irreducible in $\Q[x]$. 

It remains to prove the claim of what the ring of integers is. 
	\[
	\Disc(\Z[\zeta]/\Z)= \pm \Nm{L/\Q}(\Phi_{p^n}'(\zeta))
	\]
Now using the quotient rule,
	\[
	\Phi_{p^n}'(x)= \dfrac{p^n x^{p^{n}-1}(x^{p^{n-1}}-1) - (x^{p^{n}}-1) p^{n-1} x^{p^{n-1}-1}}{(x^{p^{n-1}} -1)^2}
	\]
But then we have
	\[
	\Phi_{p^n}'(\zeta)=\dfrac{p^n\zeta^{p^n-1}(\zeta^{p^{n-1}}-1) - 0 \cdot p^{n-1} \zeta^{p^{n-1}-1}}{(\zeta^{p^{n-1}}-1)^2}= \dfrac{p^n \zeta^{p^n -1}}{(\zeta^{p^{n-1}}-1)}
	\]
The denominator is an integer so the norm of the denominator is an integer. The norm of the numerator is a power of $p$ times a power $\pm 1$ (as the norm of $\zeta$ is $\pm 1$). Therefore, we have shown $\Nm{L/\Q} \Phi_{p^n}(\zeta)$ is a power of $p$. Then $\Disc(\Z[\zeta]/\Z)$ is a power of $p$ so when you look at the dual basis, the index of $\Z[\zeta]$ in the $\Z$-module generated by the dual basis are also a power of $p$. The $\Z$-module generated by the dual basis contains $\co_L$: $\Z[\zeta] \subseteq \co_L \subseteq Z$, the $\Z$-module generated by the dual basis of $\{1,\ldots,\zeta^{\phi(p^n)-1}\}$ and the index of $\Z[\zeta]$ in $Z$ is a power of $p$. So if the integers are not $\Z[\zeta]$, the contain $\Z[\zeta]$ with an index a power of $p$. To show $\co_L=\Z[\zeta]$, it is enough to show that there is no element of $\Z[\zeta]$ that is not $p$ times an element that's actually in $\co_L$. Meaning, it is enough to show $\frac{1}{p} \sum_{i=0}^{\phi(p)-1} b_i \zeta^i$ is not in $\co_L$ if all $b_i \in \Z$ but some $b_i \notin p\Z$. [If $\co_L \neq \Z[\zeta]$, then $[\co_L:\Z[\zeta]]$ is a power of $p$. Then there is at least some element of $\Z[\zeta]$ which is $p$ times an integer (that is something in $\co_L$) but not $p$ times something in the subring $\Z[\zeta]$.]


The $\Z$-module generated by $1,(1-\zeta),(1-\zeta)^2,\ldots,(1-\zeta)^{\phi(p^n)-1}$ is the same as $\Z[\zeta]=\Z[1-\zeta]$. 
	\[
	\dfrac{1}{p} \sum_{i=0}^{\phi(p^n)-1} b_i \zeta^i= \dfrac{1}{p} \sum_{i=0}^{\phi(p^n)-1} c_i (1-\zeta)^i
	\]
We are asking if this is in $\co_L$, where some of the $c_i$ are in $\Z$ but not in $p\Z$. Let $i_0$ be the smallest $i$ with $c_{i_0} \notin p \Z$. Then we have
	\[
	\dfrac{1}{p} c_{i_0} (1-\zeta)^{i_0} + \dfrac{1}{p} \sum_{i>i_0} c_i (1-\zeta)^i \in \co_L
	\]
We can multiply by powers of $1-\zeta$ and stay in $\co_L$. Multiply by $(1-\zeta)^{\phi(p^n)-1-i_0}$. 
	\[
	\dfrac{1}{p} c_{i_0} (1-\zeta)^{\phi(p^n)-1} + \dfrac{1}{p} \sum_{i>i_0} c_i (1-\zeta)^{\phi(p^n)-1+(i-i_0)} \in \co_L
	\]
The exponents $\phi(p^n)-1+(i-i_0)$ are all at least $\phi(p^n)$ and $(1-\zeta)^{\phi(p^n)} \Z[\zeta]=p\Z[\zeta]$. Therefore, the term on the right is in $\co_L$ (note that the sum is clearly in $p\Z[\zeta]$ and the $\frac{1}{p}$ cancels the $p$). Then the term on the left must be in $\co_L$. But $p \nmid c_{i_0}$. Now if this left term was an integer, its trace and norm have to be a rational integer. We calculate the norm of $\delta:=\frac{1}{p} c_{i_0} (1-\zeta)^{\phi(p^n)-1}$.
	\[
	\Nm{L/\Q}(\delta)= \left(\dfrac{c_{i_0}}{p}\right)^{\phi(p^n)} \Nm{L/\Q}(1-\zeta)^{\phi(p^n)-1}
	\]
Now $\Nm{L/\Q}(1-\zeta)=\prod_{\sigma: L \hookrightarrow \overline{\Q}} (1-\sigma(\zeta))$ (since we have a finite separable extension) but as  $\Phi_{p^n}=\prod_{\sigma: L \hookrightarrow \overline{\Q}} (x-\sigma(\zeta))$, since $\Phi_{p^n}(x)$ is irreducible, we must have
	\[
	\Nm{L/\Q}(1-\zeta)=\prod_{\sigma: L \hookrightarrow \overline{\Q}} (1-\sigma(\zeta))= \Phi_{p^n}(1)=p
	\] 
But then looking at the powers of $p$ in $\delta$, we have $\delta \notin \Z$. \qed \\

This is a general strategy: if you have a submodule of some finitely generated abelian group and you wish to prove that it is the whole group, prove that if you look at the whole group mod your subgroup you have no torsion. Now we shall say a bit more on the results on cyclotomic ring of integers:

\begin{prop}
For all $n \geq 0$, the $n$th cyclotomic polynomial 
	\[
	\Phi_m(x)=\prod_{\substack{\zeta \in \overline{\Q} \\\zeta^m=1, \\ \zeta^l \neq 1 \text{ for }l<m}} (x-\zeta)
	\]
is irreducible in $\Q[x]$. If $\zeta=\zeta_n$ is any root, then
	\begin{enumerate}[(a)]
	\item $[\Q[\zeta]:\Q]=\phi(m)$
	\item $\co_{\Q[\zeta]}=\Z[\zeta]$
	\item There is an isomorphism of groups $(\Z/m)^* \to \Gal(\Q[\zeta]/\Q)$ given by $a \mapsto \sigma_a$, where $\sigma_a(\zeta)=\zeta^a$. So in particular, $\Gal(\Q[\zeta]/\Q)$ is abelian. 
	\item $\co_L=\Z[\zeta]$ has discriminant $D(\co_L/\Z)=d_{L/\Q}$ which only involves powers dividing $m$. In otherwise, $\co_L/\Z$ is unramified outside $m$. 
	\end{enumerate}
\end{prop}


\begin{thm}[Kronecker-Weber]
If $L/\Q$ is a finite abelian extension then $L \subseteq \Q(\zeta_m)$ for some $m$.
\end{thm}


This is all related to Hilbert's 12th Problem: give a similar construction, via explicit elements, of all abelian (respectively, all) extensions of a given number field $F$. The `good' answers known are for $F=\Q$ (Kronecker-Weber), $F=\Q(\sqrt{d})$ for $0<d \in \Z$ square-free (Weber, Weierstrass, et al.) using torsion points on elliptic curves, and $F$ number field so that $F$ is a quadratic extension of a field $F^+$ and all embeddings $\sigma: F^+ \to \C$ have $\sigma(F^+) \subseteq \R$ and all $\tau: F \to$ have $\tau(F^+)\not\subseteq \R$ (complex multiplication theory due to Shimura et al.). This is related also to Stark's conjecture. 


There is a characteristic $p$-analog of cyclotomic fields called Drinfeld modules. Anytime you prove something about number fields, you should try to prove something about function fields of transcendence degree 1 over a finite field as these are very similar. Let $\zeta=\zeta_m$ a root of $\Phi_m(x)$. Consider the set $\mathcal{M}_m=\{\zeta^i \colon i \in \Z\}$. This is $\ker \{\phi_m \colon \overline{\Q}^* \to \overline{\Q}^*, \alpha \mapsto \alpha^m\}$. If you look at the field adjoining $\zeta$, it is the same as the field obtained by joining all the kernel elements of these homomorphisms: $\Q \subseteq \Q(\zeta)=\Q(\mathcal{M}_m) \subseteq \overline{\Q}$. 


Now let $L$ be a field with characteristic $p>0$. Let $\tau$ be an indeterminate. Define $L\{\tau\}$ to be the (noncommutative) ring of all twisted polynomials: $a_0+a_1 \tau + a_2\tau^2 + \cdots + a_n \tau^n$ with $a_i \in L$ and $\tau a=a^p \tau$ if $a \in L$. Take $A$ to be a commutative ring. 

\begin{dfn}[Drinfeld Module]
A Drinfeld module is a homomorphism of rings $\psi: A \to L\{\tau\}$. 
\end{dfn}

Note that if $\overline{L}$ is an algebraic closure of $L$, then $\tau$ gives a Frobenius automorphism of the additive group $L^+$ via $\tau: \overline{L}^+ \to \overline{L}^+$ given by $\beta \mapsto \beta^p=\tau(\beta)$: $\tau(\beta+\gamma)=(\beta+\gamma)^p=\beta^p+\gamma^p=\tau(\beta)+\tau(\gamma)$. Each element $\lambda$ of $F\{\tau\}$ gives an endomorphism $L^+ \to L^+$. 


\begin{dfn}[Carlitz Module]
Take $L=\F_p(T)$, $A=\F_p[T] \ma{\psi} L\{\tau\}$ a ring homomorphism determined completely by $\psi(T)$ and choose $T \mapsto T+\tau$. This gives an action of $A$ on $L$. The resulting module is called the Carlitz Module. Note we require $\psi(A) \not\subseteq L$.
\end{dfn}


\begin{dfn}[Division Points]
Suppose $\pi(t)$ is any nonzero polynomial in $A=\F_p[T]$. Let $\mu_{\pi(T)}=\{ \beta \in \overline{L}^+ \colon \psi(\pi(T))(\beta)=0\}$. That is, $\mu_{\pi(T)}$ is the $\pi(T)$ division points on $\overline{L}^+$ with respect to the $\F_p[T]$ $A$ module structure of $\overline{L}^+$ coming from the Carlitz module. 
\end{dfn}

\begin{thm}[Carlitz] \hfill
\begin{enumerate}[(a)]
\item $L(\mu_{\pi(T)})$ is an abelian extension of $L=\F_p(T)$.
\item $\Gal(L(\mu_{\pi(T)})/L) \stackrel{\sim}{\longleftarrow} (A/\pi(T)A)^*$.
\end{enumerate}
\end{thm}

For (b), we know that $\sigma \in \Gal$ is determined by its action on $\mu_{\pi(T)}$. On the other hand, take $a \in A$ and let it act on the division points $\mu_{\pi(T)}$. Taking this element and modifying it by $\pi(T)A$, $\pi(T)$ sends all division points to 0 so it does not move them around at all. This is a great counterpart to the analogy for $\Q(\mu_m)$: $\Q(\mu_m)/\Q$, $\mu_m=\ker( \overline{\Q}^* \to \overline{\Q}^*)$ given by $\alpha \mapsto \alpha^m$, where $m$ plays the role of $\pi(T)$ [$\psi(\pi(T)): \overline{L}^+ \to \overline{L}^+$]. Then $\mu_m$ is the `stuff' mapped to the identity and $\pi(T)$ is the `stuff' mapped to the identity in the Carlitz case. [Note here $\Z \sim A$.] 


\begin{ex}
Take $\pi(T)=T$. Now $\psi(\pi(T))=T+\tau \in L\{\tau\}$, where $L=\F_q(T)$. This acts on $\overline{L}^+$ via $\beta \mapsto \psi(\pi(T))(\beta)=(T+\tau)(\beta)=T\beta+\beta^p$, where $\beta \in \overline{L}^+$. Now
	\[
	\mu_T=\{ \beta\in \overline{L}^+ \colon \psi(T)(\beta)=\beta T+\beta^p=0\} =\{0\} \cup \{\beta \in \overline{L}^+ \colon \beta^{p-1}=T\}
	\]
Furthermore, we have $L(\mu_T)=\F_p(T^{1/(p-1)})$---all the other roots differ by a $(p-1)$-st root of unity. The $(p-1)$-st roots of unity are in the right side so that $L(\mu_T)$ is a Kummer extension. 
	\[
	\begin{tikzcd}
	\Gal(L(\mu_T)/L) \isoarrow{d} \arrow{r}{\psi} & L\{\tau\} \\
	\F_p^*=(\F_p[T]/T\F_p[T])^* \arrow{r} & T+ \tau
	\end{tikzcd}
	\]
Now 
	\[
	\begin{tikzcd}
	L(\mu_T)=\F_p(T^{1/(p-1)}) \arrow[dash]{d} \\
	\F_p[T] \subseteq L=\F_p(T)
	\end{tikzcd}
	\]

For more on this, see Rosen \emph{Number Theory in Function Fields}.
\end{ex}

\begin{dfn}
A Dedekind ring is an integral domain $A$ such that $A$ is noetherian, integrally closed, and every nonzero prime ideal is a maximal ideal, and there is at least one non-zero prime ideal.
\end{dfn}

Note that requiring one non-zero prime is to exclude fields (forcing dimension 1). Many texts such as Lang, Samuel, Jansz, etc. exclude this condition (so that fields are Dedekind). Note that Hartshorne requires this last condition. 

\begin{dfn}[Fractional Ideal]
A fractional ideal of $A$ is a nonzero finitely generated $A$-submodule of $F$. 
\end{dfn}

We can multiply such ideals, say $I,J$, by taking the $A$-submodule generated by all products $\alpha \beta$ with $\alpha \in I$ and $\beta \in J$.

\begin{thm}
Every fractional ideal $I$ in a Dedekind ring can be written uniquely as a product of integral powers of primes: $P_1^{b_1}P_2^{b_2} \cdots P_n^{b_n}$, where $P_i$ are distinct primes and $Pi^{-1}=\{ \beta \in F \colon \beta P_i \subseteq A\}$. The fractional ideals form a commutating group, $\I(A)$, with subgroup of principal fractional ideals $\Prin(A)=\{A \beta \colon 0 \neq \beta \in F^\times\}$. 
\end{thm}

We will be interested in $\Cl(A)=\I(A)/\Prin(A)$.
