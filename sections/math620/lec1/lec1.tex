% !TEX root = ../../../alg_num_theory.tex
% 9 - 09 - 2009
\newpage
\subsection{Algebraic Numbers \& Integers\label{sec:620_1}}

\begin{dfn}[Algebraic Number/Integer]
$\alpha \in \C$ is an algebraic number (respectively, algebraic integer) if $\alpha$ is the root of an equation
	\[
	x^n + a_{n-1} x^{n-1} + \cdots + a_0= 0
	\] 
for some $n \geq 1$ and some $a_i \in \Q$ (respectively, $a_i \in \Z$). If $\alpha$ is not algebraic, we say that $\alpha$ is transcendental. 
\end{dfn}

\begin{ex} \hfill
\begin{enumerate}[(i)]
\item $\sqrt{2}$ is an algebraic integer as it satisfies $x^2-2=0$.
\item $\frac{\sqrt{2}}{2}$ is an algebraic number but not an algebraic integer.
\item $\alpha=e^{2\pi i/n}$, the cyclotomic integers, is an algebraic integer: $x^n-1=0$.
\end{enumerate}
\end{ex}

\tb The number of algebraic numbers is countable (the countable union of countable sets is countable). So `most' numbers are transcendental. However, how can you tell if a particular number is algebraic or transcendental? 

\begin{thm}[Liouville]
Suppose $\alpha \in \R$ is algebraic of degree $n>1$ so that
	\[
	\alpha^n + a_{n-1} \alpha^{n-1} + \cdots + a_0 = 0
	\]
for some $a_i \in \Q$. Then for all constants $C>0$ and $\ep >0$, there are only finitely many rationals $p/q$ with $p,q \in \Z$ so that
	\[
	\left| \alpha - \dfrac{p}{q} \right| < \dfrac{C}{q^{nt+\ep}}
	\]
\end{thm}

\tb Moral: Algebraic numbers are hard to approximate using rationals. If $\alpha$ does have infinitely many `good' rational approximations, then it is transcendental. \\

\tb This theorem is great at producing transcendental numbers and is the moral behind most transcendence proofs. 

\begin{ex}
The Liouville number: $\alpha= \sum 10^{-n!}$. Look at $p_i/q_i=\dfrac{\sum_{n=1}^i 10^{i!-n!}}{10^{i!}}$. 
	\[
	|\alpha - p_i/q_i| = \sum_{n>i}^\infty \dfrac{1}{10^{n!}} 
	\]
Generally, take an integer $b \geq 2$ and sequence $(a_1,a_2,\ldots)$ with $a_k \in \{0,\ldots,b-1\}$ with infinitely many nonzero and then $x= \sum_{k=1}^\infty a_k/b^{k!}$. The above is the special case of $b=10$ and $a_k=1$. Generally, letting $q_n=b^{n!}$ and $p_n=q_n \sum_{k=1}^n a_k/b^{k!}$
	\[
	0<\left| x- \dfrac{p_n}{q_n}\right| = \sum_{k=n+1}^\infty \dfrac{a_k}{b^{k!}} \leq \sum_{k=n+1}^\infty \dfrac{b-1}{b^{k!}} < \sum_{k=(n+1)!}^\infty \dfrac{b-1}{b^k} = \dfrac{b-1}{b^{(n+1)!}} \sum_{k=0}^\infty \dfrac{1}{b^k} \leq \dfrac{b-1}{b^{(n+1)!}} \cdot \dfrac{b}{b-1}= \dfrac{b}{b^{(n+1)!}} \leq \dfrac{b^{n!}}{b^(n+1)!} = \dfrac{1}{q_n^n}
	\]
where we have used $n \cdot n!=n \cdot n! + n! - n! = (n+1)!-n!$. In particular, all these numbers are irrational and transcendental. Generally, a Liouville number is an irrational number $\alpha$ with the property that for each positive integer $n$, there are $p,q$, with $q>1$ such that $0<|x-p/q|< 1/q^n$.
\end{ex}

\begin{thm}[Hermite, 1878]
$e$ is transcendental. 
\end{thm}

\begin{thm}[Lindemann, 1882]
$\pi$ is transcendental. 
\end{thm}

A good reference for these proofs is Hardy and Wright ``Theory of Numbers''. 

\begin{thm}[Roth's Theorem, 1952]
If $\alpha$ is algebraic of degree $>1$ and if $\ep>0$, 
	\[
	\left| \alpha - \dfrac{p}{q} \right| < \dfrac{1}{q^{2+\ep}}
	\]
has finitely many solutions in integer $p,q$.
\end{thm}

\tb This was first conjectured by Siegel from work by Thue. The proof is ineffective in that it gives no bounds on $p,q$. Roth won the fields medal because of this work. Roth was supposed to pass his qualifying exams but was a nervous person. So he was told to take a practice test (which turned out to be the real test). He also got into many arguments with Serge Lang and once wrote a terrible book review for him. \\

\begin{thm}[Gelfond-Schneider, 1934]
If $\alpha,\beta$ are algebraic with $\alpha \notin \{0,1\}$ and $\beta$ is not rational, then $\alpha^\beta$ is transcendental. 
\end{thm}

\tb We need that $a,b$ be algebraic as $(\sqrt{2}^{\sqrt{2}})^{\sqrt{2}}=2$. In fact, $2^{\sqrt{2}}$ is called the Gelfond-Schneider constant. Gelfond's constant is $e^\pi$ as well as $i^i$. Take $(-1)^i=(e^{\pi i})^i = e^{\pi}$. \\

\begin{thm}[Baker]
If $\beta_1,\beta_2,\alpha_1,\alpha_2$ are algebraic, then either $\beta_1!\log \alpha_1 + \beta_2 \log \alpha_2=0$ or there is a compatible lower bound for $|\beta_1\log\alpha_1+\beta_2\log\alpha_2|$ in terms of the sizes of the coefficients in the equation for the $\alpha_i$ and $\beta_j$. 
\end{thm}

\tb Baker also won the Fields Medal. This result later was used to prove the class number 1 problem. Stark later gave another solution to the class number 1 problem. But both later learned that an Electrical Engineer named Heeger first solved it but no one believed the proof (it was written oddly). 

\begin{thm}[Ap\'ery]
$\zeta(3)$ is irrational. 
\end{thm}

\tb $\zeta(3)$ is Ap\'ery's constant. $\zeta(3) \approx 1.202$. This is conjectured to be transcendental. For even integers, $\zeta(n)$ is known to be a rational multiple of a power of $\pi$. 

\tb Integrality generalizes the notion of an algebraic number.

\begin{dfn}[Integral]
If $A \subseteq R$ are commutative rings (which are always assumed to have identity), then $x \in R$ is said to be integral over $A$ if it satisfies a monic polynomial with coefficients in $A$. 
\end{dfn}

\begin{ex}\hfill
\begin{enumerate}[(i)]
\item $A=\Q$ and $R=\C$: $\alpha$ is integral over $\Q$ if and only if $\alpha$ is an algebraic number.
\item $A=\Z$ and $R=\C$: $\alpha$ is integral over $\Q$ if and only if $\alpha$ is an algebraic integer.
\item $A=F[k]$ and $R=F[u,v]/F[u,v]f$, where $f=v^n+a_{n-1}v^{n-1}+\cdots a_0$ with $a_i \in A$. Then $R$ is the affine ring of the plane curve defined by setting $f=f(u,v)=0$. Then $x=v$ is integral over $A$.
\end{enumerate}
\end{ex}

\begin{thm}
Suppose $A \subseteq R$ commutative rings and $x \in R$. The following are equivalent:
\begin{enumerate}[(i)]
\item $x$ is integral over $A$
\item the subring $\langle A,x \rangle$ of $R$ is finitely generated $A$-module. 
\item There is a subring $B\subseteq R$ such that $A \subseteq B$, $x \in B$, and $B$ is a finitely generated $A$-module. 
\end{enumerate}
\end{thm}

\pf
\begin{enumerate}
\item[$1 \to 2$] If $x^n+a_{n-1}x^{n-1}+\cdots+a_0=0$. Then $\langle A,x\rangle$ is generated as an $A$-module by $1,x,\ldots,x^{n-1}$. [Need identity]. 
\item[$2 \to 3$] Obvious, $B=\langle A,x\rangle$.
\item[$3 \to 1$] Suppose $B$ as given in the theorem statement. Suppose $B$ is generated by $w_1,\ldots,w_n$ as an $A$--submodule of $R$, i.e. $B=Aw_1 + \cdots + A w_n \subseteq R$. Now $xB \subseteq B$. Therefore,
	\[
	\begin{split}
	xw_1&= a_{1,1}w_1 + \cdots + a_{1,n} w_n \\
	xw_2&=a_{2,1}w_1 + \cdots + a_{2,n} w_n \\
	       \vdots &= \vdots \\
	xw_n &= a_{n,1} w_1 + \cdots + a_{n,n} w_n      
	\end{split}
	\]
Writing this in matrix form, we have $M=xI_n - (a_{ij})$ and $M m_j=0$, where $m_j$ is the vector with 0's everywhere except the $j$th spot. Therefore, we have $\text{cof}(M)^T M=\det M I_n$ kills $B$: wow $\det M \in R$ and $\det M \cdot w_j =0$. But $1 \in B$! Then $\det M=0$ so that $x$ satisfies the polynomial equation given by the determinant. This resulting polynomial is degree $n$ and is monic. \qed \\
\end{enumerate}

\tb This final criterion is very useful as the proof gives an algorithm for finding the equation the element $x$ satisfies. Note then any finitely generated commutative ring containing the given commutative ring is integral over the ring. 

\tb Above, we only need commutative, not even integral domains. The rings could have zero divisors or nilpotents. What happens if $R$ is not commutative? What happens in the above cases? What if $R$ is not commutative but $A \subseteq Z(R)$?

\begin{cor}
If $A \subseteq R$ are commutative rings, the set $A'$ of $x \in R$ which are integral over $A$ is a subring of $R$. The subring $A'$ is called the integral closure of $A$ in $R$.
\end{cor}

\pf Say $x,y \in A'$ and $B_x=\langle A,x \rangle$, $B_y=\langle A,y \rangle$. Say the polynomial equation $x,y$ satisfy are of degree $n,m$, respectively. Then $\{x^i y^j\}_{1\leq i,j \leq \max\{n,m\}}$ certainly generates everything in $\langle A,x,y \rangle$. But then taking this to be $B$, we see $A \subseteq B \subseteq R$ and $B$ is finitely generated so that $x \pm y$ and $xy$, $rx$ are all in $A'$. Then $A'$ is a subring. \qed \\

\tb We say that $R/A$ is integral if every element of $R$ is integral over $A$. 