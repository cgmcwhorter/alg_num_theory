% !TEX root = ../../../alg_num_theory.tex
% 9 - 30 - 2009
\newpage
\subsection{Applications of Dedekind Rings \& Decomposition of Primes\label{sec:620_7}}

We now shall give some context and background for Dedekind rings. First, we begin with Fermat's last theorem. Let $p>2$ be prime. Suppose $x^p+y^p=z^p$ has a solution in relatively prime integers $x,y,z \in \Z \setminus \{0\}$ and $p \nmid xyz$. Now $z^p=\prod_{k=0}^{p-1} (x+ \zeta^k y)$, where $\zeta$ is a primitive $p$th root of unity using the fact that $w^p-1 = \prod_{k=0}^{p-1} (w- \zeta^k)$. That is, if there were a solution, we would have a factorization in $\Z[\zeta]$, where $\zeta$ is a primitive $p$th roof of unity. This works if $\Z[\zeta]$ were a UFD.

Suppose $\Z[\zeta]$ is a UFD (`almost never' true, i.e. for finitely many $p$, fails first for $p=23$). Consider the irreducible factorizations of the $x+\zeta^k y$. 
\begin{enumerate}[(i)]
\item $\{x+\zeta^ky\}_{k=0}^{p-1}$ are pairwise relatively prime: $(x+\zeta^ky)-(x+\zeta^ly)=(\zeta^k-\zeta^l)y$. Since $\zeta^k-\zeta^l$ has norm $\pm p$, the only factors on the right are multiples of $p$ or divisors of $y$. But $y$ is relatively prime to $p$. Then if they were not relatively prime, a divisor would divide either the difference or $y$. Then one can routinely show that $x$ and $y$ are not relatively prime. 

\item Now if we are in a UFD and these are relatively prime, the product $\prod_{k=0}^{p-1} (x+ \zeta^k y)$ would have to be a $p$-power. Then all the factors in the product are $p$-powers times a unit. That is for all $k$, $x+\zeta^ky= \epsilon_k \alpha_k^p$ for some unit $\epsilon_k \in \Z[\zeta]^\times$ and some $\alpha_k \in \Z[\zeta]$.

\item Now one must show that $\Z[\zeta]^\times=\Z[\zeta+\zeta^{-1}]^\times \cdot \langle \zeta \rangle$. Morally what one is saying is that taking the field generated by a primitive $p$th root of unity, there is a totally real subfield such that almost all the units come from this subfield. So the units coming up in the above equations are coming up from small subrings. 
	\[
	\begin{tikzcd}
	\Q(\zeta) \arrow[draw=none]{r}[sloped,auto=false]{\supseteq} \arrow[dash]{d} & \Z[\zeta] \arrow[dash]{d} \\
	\Q(\zeta+\zeta^{-1}) \arrow[draw=none]{r}[sloped,auto=false]{\supseteq} \arrow[dash]{d} & \Z[\zeta+\zeta^{-1}] \arrow[dash]{d}\\
	\Q \arrow[draw=none]{r}[sloped,auto=false]{\supseteq} & \Z
	\end{tikzcd}
	\]

\item Looking at $x^p$ in $\Z[\zeta]/p\Z[\zeta]$, we have $x^p= (a_0+a_1\zeta_1+\cdots+a_{p-1}\zeta^{p-1})^p \mod p$ is $a_0^p + a_1^p \zeta^p + \cdots+a_{p-2}\zeta^p \in \Z$. 

\item  $(x+\zeta^k y)=\epsilon_k \alpha_k^p \mod p$ so that $(x+\zeta^ky) \mod p$ is in the image of $\{1,\ldots,\zeta^{p-1}\} \Z[\zeta+\zeta^{-1}]$ in $\Z[\zeta]/p=(\Z/p) \oplus (\Z/p)\zeta \oplus \cdots \oplus (\Z/p)\zeta^{p-2}$. Vary the $k$ and show this does not happen. 
\end{enumerate}

However, $\Z[\zeta]$ is not generally a UFD! The question is how bad is the factorization then? This is exactly where fractional ideals came into play. You do not have factorization with elements but you do with ideals. This leads to the study of class groups. 


As a second context for Dedekind rings, we look at curves over a field $k$. Say $F$ is a field of transcendence degree 1 (a finite (algebraic) extension of $k(t)$ for any $t \in F\setminus k$). For example, take $F=k(t)$. Note that Weil showed one can always find $t$ with $F$ separable, finite (algebraic) over $k(t)$. Then $F$ defines a unique regular projective $C_F$ over $k$. The affine rings of $C_F$ are Dedekind $k$-algebras $A \subseteq F$ with $A$-finitely generated over $k$ and fraction field $F$. For example, if $F=k(t)$ then $A=k[t]$ or $A=k[t^{-1}]$ or $A=k[t,t^{-1}]$. Now $\Spec A=\{p\subseteq A \colon p \text{ prime}\}=\{(0)\} \cup \{p \text{ maximal}\}$. For example, $\Spec k[t]=\{(0)\} \cup \{k[t]\pi(t) \colon \pi \text{ irred, monic}\}$. Now to glue, if $A,A'$ are affine rings of $F$, then $A\cdot A'$ is also an affine ring. Then we have a diagram 
	\[
	\begin{tikzcd}
	\Spec A & & \Spec A' & & Q \cap A & & Q \cap A' \\
	& \Spec(AA') \arrow[hook]{ul} \arrow[hook]{ur} & & & & Q \arrow{ul} \arrow{ur} &
	\end{tikzcd}
	\]
where the inclusion is taking a prime ideal of $A\cdot A'$, say $Q$, and identify this with the prime $Q \cap A$, $Q \cap A'$, respectively. [Note, $A \cdot A'$ is finitely generated over $k$ as an algebra.] It turns out that these inclusions have finite complement, i.e. there are only finitely many primes missed by the inclusion. We wish to identify these `points', i.e. these primes. The points of $C_F$ are the `union' of all these spectra glued over diagrams of the form above; that is, $C_F= \varinjlim_{\text{A affine ring}} \Spec A$. If you glue $\Spec(k[t]) \cong \A_k$ and $\Spec(k[t^{-1}]) \cong \A_k$ over $\Spec(k[t,t^{-1}] \cong \A_k \setminus \{tk[1]\}$, one obtains $C_{k(t)} = \P^1_k$.  



Now define $\Div(C_F)$ to be the free abelian group on $[P]$ as $P$ ranges over all maximal ideals of affine rings $A$ of $F$, identified with one another via diagrams of the form above. Certainly, this `includes' the fractional ideals of $A$, $I_F(A)$: choosing an affine ring, one looks at the fractional ideal which factors as $P_1^{a_1} \cdots P_s^{a_s}$ and map it to $\sum_i a_i [P_i]$. Hence inside $\Div(C_F)$, we have the group of fractional ideals of every Dedekind subring (every affine piece). Generally, $\Div(C_F)$ is bigger since not every $P$ coming from some affine ring necessarily comes from the affine ring $I_F(A)$. Furthermore inside $\Div(C_F)$, one has $\Prin(C_F):=\{\div(f) \colon f \in F\}$, where $\div f= \sum_{P \text{ max}} n_P P$ and $n_P$ is such that for $A$ Dedekind with $f \in A$ and $F=\Frac(A)$ (one can take integral closures for this) so that $fA=P^{n_p}$ (product of integral powers of primes of $A$ different from $P$). We can define a map $\Pic(C_F):=\Div(C_F)/\Prin(C_F) \to \Cl(A)=\text{IF}(A)/\Prin(A)$ via mapping $C_F \in [Q]$ (for some $Q$ a maximal ideal in an affine ring) to 0 if $Q$ does not come from $A$ and $[Q]$ otherwise. In fact, this mapping is surjective: $\Cl(A)$ is the quotient of $\Pic(C_F)$ by the subgroup generated by the `subgroups that do not come from $A$.' So an interesting fact coming from this is that if $k$ is algebraically closed, we obtain the following theorem:

\begin{thm}[Jacobi, Weil, $\ldots$]
There is an abelian variety $\Jac(C)/k$ (projective, regular, group variety) such that $\Jac(C)(k)=\Div(C_F)/\Prin(C_F)$.
\end{thm}

As an example, take $F=k(t)(y)$, where $y^2=h(t) \in k[t]$ is a cubic without multiple roots (assuming $k=\overline{k}$ and $\char k=0$ to avoid worrying about inseparability). 
	\[
	\begin{tikzcd}
	A=k[t]+k[t]y \arrow[draw=none]{r}[sloped,auto=false]{\subseteq} \arrow[dash]{d} & F \\
	k[t] 
	\end{tikzcd}
	\]
Now $F$ is an affine ring of $C_F$ and $E: y^2=h(t)$ defines an affine curve. If one looks at the `projectivization' of the curve in $\P^2$, instead of $(t,y)$, we have $(t/x,y/x)$ so that in projective space we have projective curve $\Jac(C_F) \subseteq \P^2_k=\{(x:t:y) \colon x,t,y \in k \text{ not all }0\}$, assuming the normal identifications of these triples. A good calculation is to compute $\Pic(\Jac(C_F))$. One shows the divisor group of $\Jac(C_F)$ is generated by the prime ideals of the affine ring together with one more point---the point at infinity. 
	\[
	\begin{tikzcd}
	\phantom{x} & \Div(\Jac(C_F))/\Prin(C_F) = \Pic(C_F) \arrow{d} \\
	\Pic^0(C_F)= \Div^0(\Jac(C_F))/\Prin(\Jac(C_F))  \arrow[leftrightarrow]{ur}{\sim} \arrow[swap]{r}{\sim} & E(k)
	\end{tikzcd}
	\]
So the theorem generalizes the discussion at the end of the previous lecture. In summation, Dedekind rings are a way to understand curves. 


Returning to Dedekind rings, we want to prove the statement made about decomposition of prime ideals in finite separable extension fields. We have the usual situation when $A$ is Dedekind and $L/F$ is a finite separable extension ($A'$ being the integral closure of $A$ in $L$):
	\[
	\begin{tikzcd}
	A' \arrow[draw=none]{r}[sloped,auto=false]{\subseteq} \arrow[dash]{d} & L \arrow[dash]{d}\\
	A \arrow[draw=none]{r}[sloped,auto=false]{\subseteq} & F=\Frac(A)
	\end{tikzcd}
	\]
The ideal generated by $P$, $PA'$, has a decomposition $PA'=P_1^{e_1}\cdots P_s^{e_s}$, where $e_i>0$ and distinct primes $P_i$ of $A$. The numbers $e_i=e(P_i/P)$ are the ramification degree, $f_i=f(P_i/P)=[(A'/P_i) \colon (A/P)]$ is the residue field degree over $F$, and norm $\Nm{L/F}(P_i):= P^{f_i}$. The main theorem we want to prove is the following:

\begin{thm}
If $A$ is a Dedekind ring, $F=\Frac(A)$, $L/F$ is a finite separable extension, $A'$ is the integral closure of $A$ in $L$, and $P=P_1^{e_1}\cdots P_s^{e_s} \in \Spec A$, then
	\[
	\sum_{i=1}^s e_if_i = [L:F]=\dim_{A/P}(A'/PA')
	\]
\end{thm}

We shall prove this (rather finish the proof) in the next lecture. But consider a special case as an example. In the case $\Z \subseteq \Q$, consider quadratic extensions, i.e. $[L:F]=2$. There are only a few possibilities. First, $PA'=Q^2$, i.e. $P$ ramifies. Second, $PA'=Q_1Q_2$, where $f_1=f_2=1$, i.e. $P$ splits. Lastly, $PA'=Q$ in which case we say $P$ is inert ($f(Q/P)=2$). We say that $P$ splits in $A'$ if all the $P_i$ and $f_1$ are 1; that is, $P=P_1 \cdots P_{[L:F]}$. Later when the Chebotarev Density Theorem is covered, we shall see that in the case where $L$ and $F$ are number fields, there are infinitely many such $P$ in $A$.

