% !TEX root = ../../../alg_num_theory.tex
% 9 - 16 - 2009
\newpage
\subsection{Norms/Traces, Integral Closures, Dual Bases\label{sec:620_3}}

Recall the situation: $A$ is a noetherian integral domain and $A$ is integrally closed in $F=\Frac A$. Let $L/F$ be a finite separable extension of fields. Let $A'$ denote the integral closure of $A$ in $L$. 
	\[
	\begin{tikzcd}
	 A' \arrow[draw=none]{d}[sloped,auto=false]{\supseteq} \arrow[draw=none]{r}[sloped,auto=false]{\subseteq} & L \arrow[dash]{d} \\
	A \arrow[draw=none]{r}[sloped,auto=false]{\subseteq} & F=\Frac A
	\end{tikzcd}
	\]
We want to\dots
	\begin{enumerate}[(a)]
	\item Show that $A'$ is finitely generated as an $A$-module.
	\item Find a generator for $A'$ as an $A$-module.
	\item Look at the ring theoretic properties of $A$'. 
	\end{enumerate}
Taking $A=\Z$, $F=\Q$, and $L$ a number field, we want to understand $A'=\co_L$, the ring of integers in $L$. We do know that this is the unique maximal subring of $L$ that is a finitely generated abelian group. Take $\zeta_p=e^{2i\pi/p} \in \C$ a $p$th root of unity, where $p$ is prime. Consider $L=\Q(\zeta_p+\zeta_p^{-1})$. Clearly, $L/Q$ is an extension. It turns out $\co_L=\Z[\zeta_p+\zeta_p^{-1}]$. For any number field $L$, the class number $h_L$ is the number of isomorphism classes of nonzero $\co_L$-ideals. Now if the class number is 1, all the ideals are isomorphic. But the ring itself is finitely generated, in fact generated by a single element. Therefore, $h_L=1$ if and only if $\co_L$ is a PID. Now if $L$ is as given, the (Kummer-)Vandiver Conjecture states that $p \nmid h_L$. This implies Fermat's Last Theorem for the prime $p$. To see also why we would care about integral closures, we look to Algebraic Geometry. Suppose $k$ is a field and $F=k(x)$, the field of rational functions of a single variable. Take a finite separable extension $L/F$ and examine rings inside $F$, say $A=k[x^{-1}]$. Subrings have an integral closure in $L$. From Algebraic Geometry, $L$ is the function field of a projective curve $C/k$. Now $k(x)$ is the function field of $\P_k^1$. Now $\P^1_k$ has two open sets, $A_0=\spec k[x]$ and $A_1=\spec k[x^{-1}]$, which are glued together. Then $C=\spec(A_1') \cup \spec(A_0')$. Then one reason to care about integral closures is that they give a way of understanding projective curves which have a given function field. \\


We return to traces and norms: take $L/F$ a finite extension.
\begin{enumerate}[(a)]
\item $\Tr{L/F}: L \to F$
\item $\Nm{L/F}: L \to F$
\item $\text{char poly}_F: L \to F[x]$ obtained considering an embedding $L \to M_n(F)$, where $n=[L:F]$ and $\alpha$ maps to the characteristic polynomial of its matrix multiplication representative. 
\end{enumerate}

\begin{cor}
If the minimal polynomial for $\alpha$ over $F$ to be $x^d+ a_{d-1} + x^{d-1} + \cdots + a_1 x+a_0$, then	
	\[
	\Tr{F(\alpha)/F}(\alpha)= -a_{d-1}, \quad \Nm{F(\alpha)/F}(\alpha)=(-1)^d a_0
	\]
or more generally,
	\[
	\Tr{L/F}(\alpha)= - \dfrac{n}{d} a_{d-1}, \quad \Nm{L/F}(\alpha)= (-1)^n a_0^{n/d}
	\]
\end{cor}

\begin{cor}
Traces and Norms respect towers of fields: $F \subseteq L \subseteq N$ is a tower of fields, then $\Tr{N/F}= \Tr{L/F} \circ \Tr{N/L}$ and $\Nm{N/F}=\Nm{L/F} \circ \Nm{N/L}$.
\end{cor} 

\begin{thm}
If $L/F$ is a finite Galois extension with Galois group $G$ and $\alpha \in L$, then
	\[
	\Tr{L/F}(\alpha)= \sum_{\sigma \in G} \sigma(\alpha), \quad \Nm{L/F}(\alpha)= \prod_{\sigma \in G} \sigma(\alpha) 
	\]
\end{thm}

\begin{thm}[Dedekind]
For all finite extensions $L/F$ of fields, the set $\{\sigma_i\}$ of embeddings of $L$ into an algebraic closure $\ov{F}$ is linearly independent over $\ov{F}$. 
\end{thm}

Therefore, if for some $a_i \in \ov{F}$, one has $\sum^d a_i \sigma_i(\beta)$ for all $\beta \in L$, then $a_i=0$ for all $i$. This degree is the separable degree: $d=[L:F]_{\text{sep}}$, the number of possible embeddings. \\

\pf Take a minimal nonzero relation (meaning $t$ minimal):
	\[
	a_1 \sigma_1 + \cdots + a_d \sigma_t \equiv 0
	\]
with each $a_i \neq 0$. We must have $t>1$ as otherwise every element of the field is zero. There then exists at least two distinct embedding of $L$ into $\ov{F}$. But then there is a $\gamma \in L$ such that, assuming without loss of generality $\sigma_1 \not\equiv \sigma_2$, $\sigma_1(\gamma) \neq \sigma_2(\gamma)$. We have $\sigma_i(\gamma\beta)=\sigma_i(\gamma)\sigma_i(\beta)$ for all $\beta \in L$. Now we have
	\begin{equation}\label{eqn:lin1}
	a_1\sigma_1(\beta) + a_2\sigma_2(\beta) + \cdots + a_t \sigma_t(\beta)=0
	\end{equation}
Now instead substitute $\gamma \beta$ for $\beta$ and using $\sigma_i(\gamma\beta)=\sigma_i(\gamma)\sigma_i(\beta)$, we have
	\begin{equation}\label{eqn:lin2}
	a_1\sigma_1(\gamma)\sigma_1(\beta) + a_2\sigma_2(\gamma)\sigma_2(\beta) + \cdots + a_t \sigma_t(\gamma)\sigma_t(\beta)=0
	\end{equation}
But then taking the `linear combination' Equation (\ref{eqn:lin2})$- \sigma_t(\gamma)$ Equation(\ref{eqn:lin1}), we have a relation in $t-1$ terms. But this contradicts the minimality of $t$ as this combination is nontrivial. \qed \\

\begin{cor}
$L/F$ is separable if and only if there exists a $\beta \in L$ such that $\Tr{L/F}(\beta) \neq 0$.
\end{cor}

\begin{cor}
$L/F$ is separable if and only if $\Tr{L/F}$ defines a non-degenerate $F$-bilinear form. 
\end{cor}

As a vector space over $F$, $L$ is isomorphic to $d$ copies of $F$. When $L/F$ is separable, there is additional structure. Figuring out the isomorphism class of $L/F$ as a vector space is trivial. However, trying to figure out the isometry class ($(L,\Tr{L/F})$ is isometric to $(V,\langle -,-\rangle)$ if there is a morphism taking one pair to the other) of the field along with the bilinear trace pairing is a nontrivial question. 

\begin{cor}
There is an $F$-vector space isomorphism $L \to \Hom_F(L,F)=$ $F$ vector space homomorphisms: $\alpha \mapsto \Tr{L/F}(\alpha \beta)$.
\end{cor}

\pf Injective as the trace is non-degenerate. The dimensions are the same. Therefore, they must be isomorphic via this map. \qed \\

Note that this is the same as giving a non-degenerate pairing: a non-degenerate pairing from a vector space cross itself is the same as giving an isomorphism from the vector space to the linear dual. 


\noindent \underline{Notation:} If $\{u_i\}_{i=1}^d$ is an $F$-basis for a separable extension $L$ of $F$, the dual basis $\{w_j \}_{j=1}^d$ satisfies 
	\[
	\Tr{L/F}(w_i w_j^*)=
	\begin{cases}
	1, & i=j \\
	0, i \neq j
	\end{cases}
	\]
That is, $\Tr{L/F}(w_iw_j^*)=\delta_{i,j}$, the Kronecker delta function. 

\begin{thm}
Suppose $A$ is an integral domain and $A$ is integrally closed in $F=\Frac A$. Let $L/F$ be a finite separable extension of fields. Let $A'$ denote the integral closure of $A$ in $L$. 
	\[
	\begin{tikzcd}
	 A' \arrow[draw=none]{d}[sloped,auto=false]{\supseteq} \arrow[draw=none]{r}[sloped,auto=false]{\subseteq} & L \arrow[dash]{d} \\
	A \arrow[draw=none]{r}[sloped,auto=false]{\subseteq} & F=\Frac A
	\end{tikzcd}
	\] 
Then there is a basis $\{w_i\}_{i=1}^d$ in $A'$ for $L/F$. For all such bases, $A' \subseteq \oplus_{j=1}^d A w_i^*$, where $\{w_i^*\}_{i=1}^d$ is the dual basis. So if $A$ is noetherian, then $A'$ is finitely generated. 
\end{thm}

\pf Suppose $0\neq \alpha \in L$, we know $\alpha$ is algebraic over $F$:
	\[
	\alpha^d + \frac{p_{n-1}}{q_{n-1}} \alpha^{n-1} + \cdots + \dfrac{p_0}{q_0}=0
	\]
for some $p_i,q_i \in A$. Clearing denominators gives
	\[
	(r\alpha)^d+ b_{n-1} (r\alpha)^{d-1} + \cdots + b_0=0
	\]
for some $r \in A$ and $b_i \in A$. So there is a basis $\{w_i\}$ contained in $A'$ for $L$ over $F$. Take any $\{w_i\}$ with this property and $\beta \in A'$. Now $\beta=\sum_{j=1}^d a_i w_i^*$ for some $a_i \in F$. We need show $a_i \in A$. We look at $\Tr{L/F}(w_i\beta)$ in two ways. Now $\beta \in A'$ and $w_i \in A'$. Therefore, $w_i \beta \in A'$. We know the trace of something in $A'$ is some multiple of the sum of the various conjugates. But if something is integral over $A$, every conjugate is integral over $A$ (by applying $\sigma_i$). Then $\Tr{L/F}(w_i\beta)$ is a sum of elements of $L$ which are integral over $A$. But then the sum must be integral over $A$ and be in $F$ (since the trace is to $F$). Therefore, this sum is in $A$ as $A$ is integrally closed. On the other hand, $\Tr{L/F}(w_i\beta)=\sum_{j=1}^d a_j \Tr{L/F}(w_iw_j^*)=a_i$ (due to the formula above). \qed \\


This leads to two natural questions: how do you find an integral basis and how do you identify $A'$ in this free module generated by the dual basis? The first is really linear algebra while the second leads to discriminants. 

\begin{thm}
Suppose $L/F$ is a finite separable extension generated by one element, $L=F(\alpha)$ (though the Primitive Element Theorem says this always can be done). Say $\alpha^n+a_{n-1}\alpha^{n-1}+\cdots+a_0=0$, where $a_i \in F$ and $f(x)=x^n+a_{n-1}x^{n-1}+\cdots+a_0 \in F[x]$. Then
	\[
	\dfrac{f(x)}{x-\alpha}= b_{n-1} x^{n-1}+ \cdots +b_0 \in L[x]
	\]
The dual basis to $\{1,\alpha,\cdots,\alpha^{n-1}\}$ is $\frac{b_0}{f'(\alpha)}$, $\frac{b_1}{f'(\alpha)}$, $\ldots$, $\frac{b_{n-1}}{f'(\alpha)}$. Then if  $\alpha$ is integral over $A$ is integrally closed in $F=\Frac A$ and $A'$ is the integral closure of $A$ in $L$, then $A'$ is the integral closure and is contained in the direct sum of the dual basis.
\end{thm}

\pf Take $\{\sigma_i\}_{i=1}^n$ the embeddings of $L$ into $\ov{F}$ over $F$ and $\alpha_i=\sigma_i(\alpha)$. We claim
	\[
	\sum_{i=1}^n \left(\frac{f(x)}{x-\alpha_i} \right) \dfrac{\alpha_i^r}{f'(\alpha_i)} = x^r
	\]
for $0 \leq r \leq n-1$. The two sides agree on $\alpha_1,\ldots,\alpha_n$. This uses the fact that $\alpha_i$ is a root of $f$ and separability. Moreover, the $\alpha_i$ are distinct and then the difference of the two sides is a polynomial of at most degree $n-1$ which vanishes at each of these values. Then the difference must be zero. If $0 \leq l \leq n-1$, the coefficient of $x^l$ in $\sum_{i=1}^n \frac{f(x)}{x-\alpha_i} \frac{\alpha_i^r}{f'(\alpha_i)}= \sum_{i=1}^n \sigma_i (b_0+b_1x+\cdots+b_{n-1}x^n) \frac{\sigma_i(\alpha^r)}{\sigma_i(f'(\alpha_i))}$ is 0 if $l \neq r$ and 1 if $l=r$. But then is just exactly $\sum_{i=1}^n \frac{\sigma_i(b_l)}{\sigma_i(f'(\alpha))}\sigma_i(\alpha^r) = \Tr{L/F}(?)$? \qed \\

\begin{ex}
Take $F=\Q$, $A=\Z$, $L=F(\alpha)$, where $\alpha$ is a root of $x^3-x-1$. So $A'$ is the ring of integers in $L$. Certainly, $\Z[\alpha] \subseteq \co_L$. We want to find the dual basis to $\{1,\alpha,\alpha^2\}$. Now $f(x)=x^3-x-1$ and its formal derivative is $f'(x)=3x^2-1$. Now divide $f(x)$ by $x-\alpha$. The quotient is $x^2+\alpha x+ (\alpha^2-1)$, the remainder is 0 as $\alpha^3-\alpha-1=0$.
Now $x^2+\alpha x+ (\alpha^2-1)=b_2x^2+b_1x+b_0$. So the dual basis to $\{1,\alpha,\alpha^2\}$ is $\frac{\alpha^2-1}{3\alpha^2-1}, \frac{\alpha}{3\alpha^2-1},\frac{1}{3\alpha^2-1}$. Then the ring of integers, $\co_L$, lies inside the $\Z$ span of these. Then this is $\frac{1}{3\alpha^2-1}h(\alpha)$, where $h$ is a polynomial with integer coefficients [to see this, take a look at the numerators]. Call this space $H$. Then $\Z[\alpha] \subseteq \co_L \subseteq H$. So we have an upper bound for the integral closure, $A'$. Now $\Nm{L/F}(3\alpha^2-1)$ is what? We know $1 \mapsto 3\alpha^2-1$ and $\alpha \mapsto 3\alpha^3-\alpha=3(\alpha+1)- \alpha = 2\alpha +3$ and $\alpha^2 \mapsto 2\alpha^2+3\alpha$. This gives matrix 
	\[
	\begin{pmatrix}
	-1 & 3 & 0 \\
	0 & 2 & 3 \\
	3 & 0 & 2 
	\end{pmatrix}
	\]
The determinant of this matrix is $23$. Then the matrix for $\frac{1}{3\alpha^2-1}$ has determinant $\frac{1}{23}$. So $\Z[\alpha]$ has index 23. Then $\co_L=\Z[\alpha]$.  
\end{ex}